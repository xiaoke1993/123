\documentclass[review,3p,times,authoryear,12pt]{elsarticle}
\usepackage{amsmath,amsthm,amssymb}
\usepackage{algorithm}
\usepackage{algorithmicx}
\usepackage{graphicx,subfigure}
\usepackage{enumerate}
\graphicspath{{figure/}}
\newtheorem{proposition}{Proposition}

\begin{document}
\begin{frontmatter}
\newpage

\title{The Stowage Stack Minimization Problem with K-Rehandle Constraint}
\author[shu]{Ning Wang}
\ead{ningwang@shu.edu.cn}

\author[shu]{Shanhui Ke}

\author[syu]{Zizhen Zhang\corref{corl}}
\ead{zhangzizhen@gmail.com}

\address[shu]{
Department of Information Management, School of Management, Shanghai University, Shanghai, China
}
\address[syu]{
School of Data and Computer Science, Sun Yat-Sen University, China
}
\cortext[corl]{Corresponding author. Fax: +852 ********, Phone: +852 ********}

\begin{abstract}
The container stowage problem concerns with suitable placements of containers on a container-ship in a multi-port voyage.
Many scholars have conducted research with the objective of minimum rehandles. This paper investigates a variant which accommodates given containers with the minimum stacks subject to K rehandles; we call it stowage stack minimization problem with K-rehandle (SSMP-KR). The variant facilitates forwarders to evaluate the value of received individual shipping orders.
In this paper, we analyze the structure of SSMP-KR and its relationship with the circle graph.
A heuristic approach is proposed to construct near-optimal solutions in a very short computational time.
Experimental results show the effect of instance parameters on solutions.
\end{abstract}

\begin{keyword}
Containership stowage planning\sep stack minimization \sep K-rehandle \sep constructive heuristic
\end{keyword}
\end{frontmatter}


\section{Introduction}
\label{sec:i}
TODO: add importance of global trade
International trade plays a important role in promoting the development of world economy, promoting national income increase and expanding employment around the world.
Meanwhile, logistics is the main link in international trade. In view of the complexity of logistics in international trade, multimodal transport can bring a great convenience. Furthermore, the perfect fit between container transport and multimodal transport makes the international trade easier and more convenient.
Container-shipping is a kind of modern transportation tool which has a lot of advantages in speed, security, and quality comparing with the break-bulk cargo ship. Over 60\% of general cargoes are transported by containership presented by \cite{zhang2008review} (TODO: cite source). This mode is deeply accepted by shipper and carrier, and is becoming the main trend of freightage in the world. It shows powerful vitality and promising prospects.

The container stowage plan is a pivotal link to containers transportation,and the objective of which is to draw a plan of loading and discharging sequence of contains for containership in \cite{zhang2008review}. Actual examples show that containership stowage plans not only influence the income of shipping company from transportation but also have direct relation to the safety of ships and freights. These problems have troubled and aroused the interests of both scholars and commercial shipping organization in many countries since 1970s. With the development of container-shipping, more and more containers are transported by sail. Up to now, no satisfying solution has development of container transportation trade. Thus, it makes higher demand of the problem.

TODO: introduce the ship's structure, and why rehandles exist.

The previous research ignores the importance of the capacity of ship and give much emphasis on the balance of ship, the number of rehandle and so on. Sometimes we need think about the importance of capacity as well. For example, if we can use each stack sufficiently, the required number of ships will be reduced and thus the cost decreases. The problem mentioned in this paper give a priority on how to reduce the use of stack with K rehandle as much as possible. Through the research on this problem, another aspect of container stowage planning is exhibited and a lot of shipping cost can be saved by using what we propose in this paper. On the other hand, it represents a novel thinking perspective to make a research about this kind of problem and it will promote the development of academic shipping optimal problem.

We put forward the stowage stack minimization problem with K rehandle constraint problem (SSMP-KR) in this paper. Literally speaking, it is a container stowage planning problem whose objection is to find out minimization stack with K rehandle constraint on a given condition. The heuristic algorithm in this paper uses the greedy rule to finish the loading and unloading operation for every container to be loaded and unloaded in each port along the voyage. Therefore, the solution we get may be the approximately optimal rather than the global optimal.

TODO: The contribution of our paper lies in three aspects. First, Second, Third.

The remainder of this paper is organized as follows: we review extant research works in Section~\ref{sec:lr} and the formal definition of our problem is provided in Section~\ref{sec:pd}. In Section~\ref{sec:algo} we present our heuristic algorithm for the problem, as well as explain the ideas behind and specific details about the algorithm. Experiments and associated analysis are illustrated in Section~\ref{sec:ea}. Concluding remarks close the paper in Section~\ref{sec:con}.

\section{Literature Review}
\label{sec:lr}
The container stowage problem can be divided into two classes: one is management and programming at the container terminal, the other is stowage plan for containership. This paper mainly focuses on the latter. And sorted by research method, this problem will be divided into six categories: simulation based on  probability, heuristic procedures, mathematical programming approach, decision support system, expert system and evolutionary computation. The following are some description about these categories of containership stowage plan.

Since 1970s, the stowage plan problem has been studied by shipping organization for commercial demand. They tried to find a good solution  which would satisfy fewer shifting and higher efficiency when loading and discharging containers. \cite{webster1970container} first tried to solve the problem by using computer. They put forward a heuristic algorithm that could create a random stowage plan. \cite{shields1984containership} introduced a computer preplanning system(CAPS) which had been used by American President Lines. The CAPS produced stowage plan by using Monte Carlo theory. Expert system is a most active research and practical field of Artificial Intelligent (AI).  \cite{dillingham1986application} introduced their researches on expert system of stowage based on rules. Each of the suggested container movements was displayed graphically as the decisions were made. Database in the system could not only accomplish the stowage plan, but also track the location of the container. It was highly interactive, the user could interrupt the system at any time when the planning proceeded and overrode or revoked suggestions that the system had made.

 \cite{scott1978loading} constructed a model, in which containers were sorted into ten classes based on some characteristics and then three heuristic rules were used to assign gradually each type of containers to slots of the ship by four steps, while the shifting problem had not been considered in their model. \cite{avriel1998stowage} presented a 0-1 binary linear programming formulation based on minimizing the number shifting in a single bay of a ship calling at a given number of ports. Further, they showed that finding the minimum number of columns for which there was a zero shifts stowage plan is equivalent to finding the coloring number of circle graphs. \cite{botter1991stowage} created a model called complete mathematical model for the stowage plan problem, which used binary decision variables to determine the containers unloading and loading sequence for each port. The objective function of this model for the stowage plan involved two important factors: the number of re-stowing along the route and the longitudinal crane movement along the quay during the container loading and unloading operation. The advantages were that the model considered every port the ship calling at as a different stage of the whole system, which made the model much more correspond to the actual cases. But too many factors were considered in the model, the dimension of formulization was too large to solve because of large number of binary variables and constraints arising in the model. Literatures have shown that binary linear programming formulation for such problems is impracticable for real life problems due to the large number of binary variables and constraints.

Hence, the formulation of many binary variables and constraints made the working-out of model quite different, so the Suspensory Heuristic procedure is developed to solve the model. Tabu search was employed to obtain an optimal solution in \cite{wilson2001container} and his model was feasible for it can change configuration step by step until each container was assigned to specific slot. The Ship Stowage Planning Problem have been addressed in \cite{Monaco2014256}. They have first provided a detailed description of the two phases in which the decision problem is usually decomposed, along with a classification scheme and a review of the most relevant related papers. The classification scheme has a twofold benefit: highlighting similarities and differences in the previously studied stowage problems; providing an useful guide for choosing modeling and solution approaches when an analogous problem has to be faced. Then they have focused on the Terminal-oriented Ship Stowage Planning Problem at a DTS terminal that, to our knowledge, has not been investigated in the literature. They have  proposed a Binary Integer Model for the minimization of the transportation times of the containers and the yard-shifts. They have developed a Tabu Search Algorithm for finding sub-optimal solutions to the problem and have solved the model by CPLEX.

What's more, \cite{ding2015stowage} developed a heuristic algorithm which can generate stowage plans with a reasonable number of shifts for such problems. The algorithm, verified by extensive computational experimentations, performs better than the Suspensory Heuristic Procedure (SH algorithm), which, to the best of our knowledge, is one of the leading heuristic algorithms for such stowage planning problem.
\cite{tierney2014complexity} investigated the complexity of stowage planning problem and showed that the capacitated k-shift problem is solvable in polynomial time for any choice of stacks and stack capacities. And some scholars paid special attention to the process of loading and unloading, \cite{malucelli2008stack} investigated stack reordering strategies aiming at minimizing the number of pop and push operations and  focused on three versions of the problem in which reordering can take place in different phases: when unloading the stack, when loading it or in both phases. \cite{ambrosino2015mip}  proposed a Mixed  Integer  Programming (MIP) heuristic aimed at determining stowage plans in circular routes for container ships so as to give support for the ship coordinator and the terminal planner.


Algorithm of evolutionary computation came from Darwin��s biologic theory of evolution. There are two or three branches at present, among which genetic algorithm is used widely. \cite{dubrovsky2002genetic}  used a GA for solving the stowage planning problem of minimizing the number of container movements. Search space is significantly reduced by a compact and efficient encoding scheme. Ship stability criteria are reflected by appropriate constraints. Simulation runs demonstrate the efficiency and flexibility of the GA-based approach.  \cite{kammarti2013evolutionary} proposed an efficient genetic algorithm which consists on selecting two chromosomes(parent) from an initially constructed population using a roulette wheel technique. Then, the two parents are combined using a one point crossover operator. Finally, a mutation operator is performed. The variant tackled in \cite{cohen2017container} involved several constraints, inspired by real-life problems and application found in the literature. Given the complexity of the problem, which belongs to the class of NP-hard problems, a novel evolutionary metaheuristic algorithm is developed and designed. Considering the ability and flexibility of Genetic Algorithm (GA). The approach is based on a two-phase procedure, one for master planning and the other for allocation of the containers into slots. GA parameters are analyzed to achieve practical and best results. The system offers stowage allocation solutions for both phases, thus offering flexibility for a wide variety of vessels and route combinations.


In \cite{Delgado2012251}, they focus on the slot planning phase of this approach and present a Constraint Programming and Integer Programming model for stowing a set of containers in a single bay section. This so-called slot planning problem is NP-hard and often involves stowing several hundred containers. Using state-of-the-art constraint solvers and modeling techniques, however, we were able to solve 90\% of 236 real instances from our industrial collaborator to optimality within 1 second. Thus, somewhat to our surprise, it is possible to solve most of these problems optimally within the time required for practical application.


There are some extended problems in this field. Different problems arising in practice in the context of storage areas organized in stacks have been surveyed by \cite{Lehnfeld2014297}. Different problem classes have been identified, namely loading, unloading, premarshalling and combined problems. To be able to categorize all these problems in a more abstract way, they developed a classification scheme which has been used to summarize complexity results and solution methods for all problem classes.  In conclusion, many problems have already been solved by exact or heuristic methods in all four considered problem classes. Nevertheless, due to the diversity of the problems, various other versions exist. For further research, a major challenge is the development of more efficient solution algorithms integrating additional constraints coming from practical applications. As we all known, the 3D Container ship Loading Plan Problem (CLPP) is an important problem that appears in seaport container terminal operations and is well known to be NP-hard. This problem consists of determining how to organize the containers in a ship in order to minimize the number of movements necessary to load and unload the container ship and the instability of the ship in each port. In \cite{Araujo201650},  the hybrid method Pareto Clustering Search (PCS) is proposed to solve the CLPP and obtain a good approximation to the Pareto Front. The PCS aims to combine metaheuristics and local search heuristics, and the intensification is performed only in promising regions. Computational results considering instances available in the literature are presented to show that PCS provides better solutions for the CLPP than a mono-objective Simulated Annealing.

A multiobjective ship stowage planning problem(SSPP) was investigated by \cite{zhang2016multiobjective}, which aims to optimize the ship stability and the number of rehandles simultaneously. They use metacentric height, list value, and trim value to measure the ship stability. Meanwhile, the number of rehandles is the sum of rehandles by yard cranes and quay cranes and all necessary rehandles at future ports. The algorithm can produce a set of nondominated solutions. Decision makers can then choose the most promising solution for practical implementation based on their experience and preferences. Extensive experiments are carried out on two groups of instances. The computational results demonstrate the effectiveness of the proposed algorithm compared to the NSGA-II and random weighted genetic algorithms, especially when it is applied in solving the six-objective SSPP.

To the best of our knowledge, apart from the SSMP-ZR, only \cite{avriel2000container} and \cite{jensen2010complexity} discussed the stowage stack minimization and its connection with the chromatic number of circle graphs when the stack height is unlimited. At present, the research tendency of the stowage planning problem is intelligent management system, which is a new stowage research direction. Intelligent containership stowage plan system is a technological innovation in contrast with traditional stowage method. It can improve the ability of decision-making management system of container shipping by means of artificial intelligence, information technology, communications and computer technology.

 In \cite{ambrosino2004stowing} the problem of stowing containers into a containership has been faced by evaluating an exact 0-1 Linear Programming model, that is not practically useful for large cases. This has been modified by an approach proposed by the authors, consisting of heuristic preprocessing and prestowing procedures that also allow the relaxation of some constraints of the exact model. The proposed approach exhibits very good performance in terms of both solution precision and computational time. Moreover, in the performance evaluation with real size cases, it also guarantees another very important maritime performance index, in term of handling operations/hour.

\cite{wilson2001container} outlined a computer system that generates good sub-optimal solutions to the stowage pre-planning problem. This is achieved through an intelligent analysis of the domain allowing the problem to be divided into sub-problems: a generalized placement strategy and a specialised placement procedure. This methodology progressively refines the arrangement of containers within the cargo-space of a container ship until each container is specifically allocated to a stowage location.

Container ship stowage problem(CSSP ) is combinatorial optimization problem with multiobjects and multiconstraints. In order to reduce computational difficulty��CSSP is decomposed two subproblems in \cite{wei2005model}, and the first one is discussed detailed. CSSP is regarded as bin-packing problem. After being modeled��the algorithm of best fit for binpacking problem is used to solve the CSSP.  A case shows that the algorithm is feasible for container ship stowage problem.

The presented hybrid GA for loading a single container in \cite{bortfeldt2001hybrid}  is particularly suitable for strongly heterogeneous containers stowage problems.  The algorithm takes account of some constraints which are relevant in practice. Its good performance and superiority to comparative methods was verified in an extensive test. The use of a time limit kept the required computing times within acceptable limits. Then, \cite{bortfeldt2003parallel} presented a parallel tabu search algorithm for the container loading problem with a single container to be loaded. According to an extensive comparative test also including heuristics from other authors, high utilizations of the container volume are already obtained with the sequential TSA. A slight improvement of these results could be achieved by the parallelization.

\cite{haghani2001model} proposed a MIP model for developing loading plans in order to minimize the time that a vessel spends at port, and the container handling cost which shifts caused by is highly influenced by the number of unproductive but necessary an unsatisfactory arrangement of containers.

\cite{avriel1998stowage} and \cite{avriel2000container} focused on stowage planning in order to minimize the number of unproductive shifts (temporary unloading and reloading of containers at a port before their destination ports in order to access containers below them for unloading). Aspects of ship's stability and other real-life constraints are not considered. A binary linear programming (LP) model is formulated. Due to the proven NP-hardness of the problem a so-called `Suspensory Heuristic������based on earlier work by \cite{avriel1993exact} ����is developed in order to solve even large problem instances. The heuristic assigns slots in a bay to containers dynamically with respect to the sequence of ports in a vessel's route.







\section{Problem Description}
\label{sec:pd}

A ship starts its journey at port 1 and sequentially visits port 2, 3, ..., $p$, 1, 2, 3, ... $p$, 1, ..., which forms a circular route.
There are standard containers to be shipped along the journey.
At each port, the containers destined to the current port are discharged, and the containers stowed at the port are loaded to the ship.
Each container $i$ is characterized by its shipping leg $O(i)\rightarrow D(i)$, indicating that the container is loaded at port $O(i)$ and discharged at port $D(i)$.
Some containers are shipped from $i$ to $j$ ($i<j$) and some containers are shipped from $i$ to $j$ ($i>j$) going across port 1. Hence, it is easy to understand that containers with $O(i)<D(i)$ and with $O(i)>D(i)$ both exist. Moreover, the ship is never empty once sets out due to the circular feature.
All the loading and discharging ports of $N$ containers are known in advance.

We assume that the ship has enough capacity to accommodate all the containers along the journey. The ship spaces are divided into several stacks. Each stack can hold at most $H$ vertically piled containers, i.e., the height limit of every stack is $H$. When $H$ is bounded, we call the SSMP-KR as the capacitated SSMP-KR (CSSMP-KR); when $H$ is sufficiently large (e.g., $H\geq N$) or unbounded, we refer to it as the uncapacitated SSMP-KR (USSMP-KR). For the CSSMP-KR, if the number of containers in a non-empty stack is less than $H$, the stack is termed as a \textit{partial} stack; otherwise it is a \textit{full} stack. The objective of the SSMP-KR is to transport all the containers using the fewest number of stacks with K rehandles allowed. The USSMP-KR is far from the real situation in the stowage planning, since stack heights of ships are restricted by operation cranes. Hence, this paper give emphasis on the CSSMP-KR and for the convenience of description, we use SSMP-KR for short in this paper.

A feasible solution to the SSMP-KR is represented by a operation set, from which we can clearly know the slot changing of each container, even when there is an overstowage. The detailed introduction will be given in the below.

We also denote the dynamic $Layout$ as the layout of the ship after one operation for one container. We can output the layout after the operation of one container is done or the layout after all the operations at one specific port.

\section{Methodology}
\label{sec:algo}
In this section, we will talk about the heuristic algorithms to the SSMP-KR, as well as the basic performance guarantee of the algorithms.

\subsection{Heuristic Algorithms for the SSMP-KR}
\label{sec:h1}

In our algorithm, two classes are introduced in order to describe the containers' characteristic and record the operation for containers. They are Container class and OPeration class, respectively. For each container, we use arraylist "ac" to store its sequence number, origin port and destination port. No matter what the operation is , we add all of operations into the arraylist "Solution" to record every container's slot changing from the shore or previous stack to the chosen stack or shore. Other important global variables are Count\_rehandle and Numofstack. The former is no greater than the given K, and the latter no longer change until the end of the shipping. What methods aren't exhibited in the below algorithm are read() and bound(). The method read() is used to read the parameters and the detailed containership information from each instance. As for the method bound(), we use it to give a upper bound for our algorithm and the rationality proved. Another important method "GetstackNum", which is used in both loading\_rehandle\_lessK and loading\_rehandle\_equalK method, is used to get the sequence number of chosen stack.

If there is an overstowage, we use arraylist $S_1$ to store partial stacks without rehandle happening, arraylist $S_2$ to store partial stacks with rehandle happening, arraylist $S_3$ to store empty stacks whether there is a rehandle or not. The priority order of them is $S_1$, $S_2$, $S_3$. If there isn't an overstowage, we use $S_0$ to store the feasible stacks. For ease of illustration, we need to define a one dimensional array called nearport to store the nearest port for every stack. For example, three containers are stored in the same stack, and their destination port are 4,5,6, respectively, then the value of nearport of this stack is 4. If a stack is empty, its value will be set to $P+1$, P is the total number of the ports along the voyage. We firstly talk about the simple condition that the allowed rehandles is full. When traversing all stacks to choose the optimal according to our rules for each container, we add non-full stacks whose nearest port is greater than or equal to the current loading container into arraylist $S_0$. For the stacks in $S_0$, we choose the one with the nearest port. Next we talk about the complex condition that a rehandle is allowed and it is a little different from the simple condition. Under this condition, $S_1$ have the highest priority. If $S_1$ is null, we choose the stack from arraylist $S_2$ which is used to store partial stacks whose nearest port is smaller than the current loading container. Both in $S_1$ and $S_2$, we choose the stack whose nearest port is smallest. If $S_2$ is null, we choose the first stack we traverse in the $S_3$ which is used to store empty stacks. So it is the loading rules above and the following is the discharging rules.

When discharging containers, we just need to unload containers from top to bottom one by one when the container's destination port is the current port if there isn't an overstowage in the stack. when an overstowage happens, we have to unload the blocking container(s), then we can unload those right ones. As for blocking containers, we reset their origin ports as current port and their destination ports remain unchanged. That's to say, the outbound containers at the current port increases.

We give out some parts of algorithms as followings:

For all the ports along the voyage, we adopt the heuristic method to dispose containers for the given input. And we can transfer the input into the output by applying the heuristic method. So the below is the main process of our algorithm.

%\floatname{algorithm}{Main procedure}
%\renewcommand{\algorithmicrequire}{\textbf{Input:}}
%\renewcommand{\algorithmicensure}{\textbf{Output:}}
%\begin{algorithm}
%\caption{The main procedure of the algorithm}
%\begin{algorithmic}[1]
%\Require the number of ports $P$, the number of containers $N$, the height limit $H$, and the number of allowed rehandles $K$.
%\Ensure The operation solution of each container and the number of stacks used in the shipping.
%\For {$p=1 \to P$}
%\State $heuristic()$
%\EndFor
%\end{algorithmic}
%\end{algorithm}
%

The following pseudo gives a simplistic explanation of heuristic method. The variables "height", "Layout" and "Numofstack" are one dimensional array, two-dimensional array and the number of accumulative used stack in the pseudo, respectively. $Unloading\_at\_port(p)$ and $Loading\_at\_port(p)$ methods provide the way how to unload the right containers and load the right containers at each port. Given a specific instance, we can get the correspondent value of "Numofstack".


%\floatname{algorithm}{heuristic() method}
%\renewcommand{\algorithmicrequire}{\textbf{Input:}}
%\renewcommand{\algorithmicensure}{\textbf{Output:}}
%\begin{algorithm}
%\caption{The  procedure of the heuristic() method}
%\begin{algorithmic}[1]
%\Require each instance
%\Ensure  the minimum number of stacks used in each instance
%\State $height \gets \  new int[UB+1]$
%\State $Layout \gets \  new Container[UB+1][H+1]$
%\State $Numofstack \gets 0$
%\For {$p=1 \to P$}
%\State $Unloading\_at\_port(p)$
%\State $Loading\_at\_port(p)$
%\State $t \gets 0$
%\For {$i=1 \to UB$}
%\If {$height[i]!=0$}
%\State $t \gets t+1$
%\EndIf
%\EndFor
%\If {$t> Numofstack$}
%\State $Numofstack \gets t$
%\EndIf
%\EndFor
%\end{algorithmic}
%\end{algorithm}

The following unloading\_at\_port() method is used to decide which container should be discharged at port p. Then we need substitute the related information of chosen container into the Unloading\_for\_container(s,t,p) method.

%\floatname{algorithm}{Unloading\_at\_port() method}
%\renewcommand{\algorithmicrequire}{\textbf{Input:}}
%\renewcommand{\algorithmicensure}{\textbf{Output:}}
%\begin{algorithm}
%\caption{The  procedure of the Unloading\_at\_port() method}
%\begin{algorithmic}[1]
%\Require the current port $p$,
%\Ensure the sequence number of stack $s$ where the loading container is
%\For {$s=1 \to UB$}
%\For {$ t=height[s] \to 1$}
%\If {$Layout[s][t].d==p$}
%\State $Unloading\_for\_container(s,t,p)$
%\EndIf
%\EndFor
%\EndFor
%\end{algorithmic}
%\end{algorithm}
\clearpage

Considering the length of pseudo code, some details are omitted. We will give a explanatory about the following pseudo code.
We first define a one-dimensional array nearport which consists of "UB+1" integer number and it is used to store the nearest destination among all containers located in each stack. If the stack is empty, then the value will be set as "P+1". The ac is an arraylist to store all containers to store all the containers before the shipping. If one container is to be unloaded and there isn't overstowage, then it will be removed from the arraylist. For each blocking container, its origin port will be reset as current port after it is removed from its location, then it will be added into ac again.

As a result of the existence of rehandle, we need sort ac before we start loading containers at port p. The sorting rules are shown as followings. We compare origin port of containers in the first place, the number smaller, the location ahead. If the origin port is identical, compare the destination port, the number larger, the location ahead. If ac is not empty, we just need to choose the first element $cn$ to load. We use loading\_for\_container method to choose the optimal stack. After loading it into the stack $st$. $st$'s height and nearport will be updated. At last, record this operation and add the operation into the arraylist Solution. Hence, we can eventually get all the operation information.

%\floatname{algorithm}{Loading\_at\_port() method}
%\renewcommand{\algorithmicrequire}{\textbf{Input:}}
%\renewcommand{\algorithmicensure}{\textbf{Output:}}
%\begin{algorithm}
%\caption{The procedure of Loading\_at\_port() method}
%\begin{algorithmic}[1]
%\Require $the current port $p$ $
%\Ensure $containers to be loaded at port p$
%\For {$s=1 \to UB$}
%\State $ nearport[s] \gets P+1$
%\For {$j=1 \to H$}
%\If {$Layout[s][j]!=null$}
%\If {$Layout[s][j].d<nearport[s]$}
%\State $nearport[s] \gets Layout[s][j].d$
%\EndIf
%\Else
%\State $break$
%\EndIf
%\EndFor
%\EndFor
%\State $Collections.sort(ac)$
%\While {$ac.size()!=0$ \textbf {and} $ac.get(0)==port$}
%\State $Container cn \gets  ac.get(0)$
%\State $ac.remove(0)$
%\State $st \gets Loading\_for\_container(cn,Layout,nearport)$
%\State $Layout[st][height[st]+1] \gets cn$
%\If {$cn.d < nearport[st]$}
%\State $nearport[st] \gets cn.d$
%\State $height[st] \gets height[st]+1$
%\State $OPeration ope \gets new Operation(cn.Seqnum,0,st)$
%\State $Solution.add(ope)$
%\EndIf
%\EndWhile
%\end{algorithmic}
%\end{algorithm}
%\clearpage

The following Loading\_for\_container() method provides us the method how to load containers and two different sub-methods are given to return the stack where the loading container belongs. The specific procedure of two sub-methods is described in the above content.

%\floatname{algorithm}{Loading\_for\_container() method}
%\renewcommand{\algorithmicrequire}{\textbf{Input:}}
%\renewcommand{\algorithmicensure}{\textbf{Output:}}
%
%\begin{algorithm}
%\caption{The procedure of Loading\_for\_container() method}
%\begin{algorithmic}[1]
%\Require $Container con$,$Container[\ ][\ ] Layout$,$int[\ ] nearport$
%\Ensure $con$,$nearport$
%\If{$Count_rehandle==K$}
%\State $return Loading_rehandle_equalK(con,nearport)$
%\Else
%\State $return Loading_rehandle_lessK(con,nearport)$
%\EndIf
%\end{algorithmic}
%\end{algorithm}
%\clearpage

The below pseudo code of $unloading\_for\_container()$ method shows us how to unload containers one by one whether there is overstowage or not. Before the unloading operation,we need input the current port, the stack and the tier where the unloading container is. If there is overstowage, we invoke "$Relocate(s,t)$" method. After processing blocking container(s), we set out to unload the right container. We firstly record the unloading operation and add it into the arraylist Solution. They remove the container from its previous location and reduce the height of stack where it is located.

%\floatname{algorithm}{Unloading\_for\_container() method}
%\renewcommand{\algorithmicrequire}{\textbf{Input:}}
%\renewcommand{\algorithmicensure}{\textbf{Output:}}
%
%\begin{algorithm}
%\caption{The procedure of the Unloading\_for\_container() method}
%\begin{algorithmic}[1]
%\Require $stack$,$tier$,$port$
%\Ensure $unloading operation$
%\For {$t=height[stack] \to tier$}
%\State $Relocate(stack,port)$
%\State $Solution ope \gets new Operation(Layout[stack][tier].Seqnum,stack,0)$
%\State $Solution.add(ope)$
%\State $Layout[stack][tier] \gets null$
%\State $height[stack]\gets height[stack]-1$
%\EndFor
%\end{algorithmic}
%\end{algorithm}


In a word, a lot of variables and methods of the concrete Unloading and Loading plan have been involved,  we don't give the correspondent pseudo here for the sake of context length. What is identical for the two procedures is that they find the optimal  strategy for each  container during the unloading and loading process according to the greedy rule made by our algorithm. As is explained in the heading of this section, we divide the loading method into two parts on the basis of the number of reached rehandles. And similarly, the unloading method are divided into two conditions whether there is overstowage. Hence, specific steps are a little different from each other between the two circumstances.

Our algorithm can give the detailed location changes of each container and output the container status at arbitrary time. That's to say, we can grasp the container stowage process before the voyage as long as we know some information in advance. Therefore, our algorithm makes a little difference and make some sense in the reality application.


\subsection{Performance Guarantee of the Algorithms}
\label{sec:p2}

In order to simplify the problem, we show that the heuristic algorithms have a performance guarantee if the value of K is zero and we regard the port number $P$ as a fixed number. That's to say, the SSMP-KR becomes SSMP-ZR under this condition.
For ease of exposition, we first give some related notations.
\begin{itemize}
\item $\mathcal{U}^*$ \& $\mathcal{C}^*$: the optimal solution to the USSMP-ZR and the SSMP-ZR instances, respectively.
%\item $\mathcal{U}^*_p$ \& $\mathcal{C}^*_p$: the number of stacks actually used at port $p$ by the optimal plan.
\item $\mathcal{U}$ \& $\mathcal{C}$: the solution to the USSMP-ZR and the SSMP-ZR instances by the algorithm, respectively.
\item $\mathcal{U}_p$ \& $\mathcal{C}_p$: the number of stacks used at port $p$ by the algorithm.
%\item $\mathcal{Z}$ \& $\mathcal{Z}_p$: a general notation to denote any solution to the USSM-ZR instances or to the CSSM-ZR instances; $\mathcal{Z}_p$ is the number of stacks in port $p$ in solution $\mathcal{Z}$
\item $N_p$: the number of containers on the ship before its departure from port $p$.
\item $v_p$: the number of loading ports that the ship has visited before it departs from port $p$. A port is called a loading port if there exists at least one container to be loaded. Clearly, $v_p \le p$, $\forall p=1,\ldots,P$.

\end{itemize}



\begin{proposition}
For the USSMP-ZR instance, the heuristic approach generates a solution in which at most $v_p$ stacks are used before the ship departs from port $p$, i.e. $\mathcal{U}_p\le v_p, \forall p=1,\ldots,P$.
\label{pro:a2}
\end{proposition}

\begin{proof}
We use an inductive proof.

\textit{Basis}: For $p=1$, without loss of generality, we assume that port 1 is a loading port. The heuristic piles up containers in one stack in a way that the containers with later destinations are placed in lower tiers. We have $\mathcal{U}_1=v_1=1$, and therefore $\mathcal{U}_1\le v_1$ holds.

\textit{Inductive step}: Suppose that before the ship departs from port $p$, there are at most $v_p$ stacks used, i.e., $\mathcal{U}_p \le v_p$. Upon the ship arrives at port $p+1$, it first unloads the containers from the ship. This process does not increase the number of stacks utilized. If there are containers to be loaded at port $p+1$, then $v_{p+1}=v_p+1$, otherwise $v_{p+1}=v_p$.

(1) If there exist some containers to be loaded, according to our heuristic, they are placed on either the extant stacks or a new blank stack (with the containers of later destinations placed lower). Hence, $\mathcal{U}_{p+1} \le \mathcal{U}_p+1 \le v_p+1=v_{p+1}$.

(2) If no container is loaded, $\mathcal{U}_{p+1} \le \mathcal{U}_p \le v_p=v_{p+1}$.

Based on the above two steps, we have $\mathcal{U}_p\le v_p, \forall p=1,\ldots,P$. Furthermore, we have
\begin{equation*}
\mathcal{U}=\max_{p=1,\ldots,P}\mathcal{U}_p\le \max_{p=1,\ldots,P} v_p=v_P\le P
\end{equation*}
\end{proof}

\begin{proposition}
The CSSMP-ZR instance has a lower bound:
\begin{equation*}
\max\limits_{p=1,\ldots,P}(\lceil\frac{N_p}{H}\rceil) \le \mathcal C^*
\end{equation*}
\label{pro:a3}
\end{proposition}

\begin{proof}
$\lceil\frac{N_p}{H}\rceil$ is the least number of stacks to accommodate all the containers at port $p$, and thus, $\max\limits_p(\lceil\frac{N_p}{H}\rceil)$ is the least number of stacks needed throughout the journey.
\end{proof}

\begin{proposition}
For the heuristic solution to the CSSMP-ZR instance, it holds
\begin{equation*}
\mathcal{C}^* \le \mathcal{C} \le \max_{p=1,\ldots,P}(\lfloor\frac{N_p}{H}\rfloor+v_p) \le \mathcal{C}^*+v_P
\end{equation*}
\label{pro:a4}
\end{proposition}

\begin{proof}
The first inequality obviously holds.

For the second inequality, $\mathcal{C}=\max\limits_p \mathcal{C}_p$. Remind that $\mathcal{C}_p$ is the number of stacks in the layout $L^C_{p,T_p}$ resulted from our algorithm. %If we understand $\mathcal C_p$ from the perspective of the two-phrase algorithm, we know that
The number of partial stacks in the layout $L^C_{p,T_p}$ is no greater than the number of stacks in the associated layout $L^U_{p,T_p}$ (c.f., Definition \ref{def:1}). The number of stacks of $L^U_{p,T_p}$, from Proposition \ref{pro:a2}, is at most $v_p$, so the number of partial stacks in $L^C_{p,T_p}$ is no greater than $v_p$. In addition, the number of full stacks in $L^C_{p,T_p}$ is no greater than $\lfloor\frac{N_p}{H}\rfloor$. Therefore, $\mathcal{C}_p\le \lfloor\frac{N_p}{H}\rfloor+v_p$ and $\mathcal{C} \le \max\limits_p(\lfloor\frac{N_p}{H}\rfloor+v_p)$ hold.

For the third inequality, $\lfloor\frac{N_p}{H}\rfloor\leq \lceil\frac{N_p}{H}\rceil$, and $\lceil\frac{N_p}{H}\rceil$ is the lower bound of the number of stacks used at port $p$ by Proposition \ref{pro:a3}. Thus, $\lfloor\frac{N_p}{H}\rfloor \le \mathcal{C}_p^*$. It then holds $\max\limits_p(\lfloor\frac{N_p}{H}\rfloor+v_p) \le \max\limits_p(\mathcal{C}_p^*+v_p) \le \max\limits_p(\mathcal{C}_p^*+v_P) = \mathcal{C}^*+v_P$.

\end{proof}


The above proposition indicates that our approximation algorithm has a constant performance guarantee if we regard the number of ports as fixed under the condition that the number of allowed rehandle is zero. In practice, the number of ports is generally much smaller than the number of stacks used along the journey. Hence, the gap between the optimal solution and our heuristic solution is relatively small, which demonstrates that our heuristic can generate promising solutions.

In view of the number of allowed rehandle we give isn't big and it's influence on the result isn't salient. So we use the $UB$ to test
the effectiveness of SSMP-KR.




\section{Experiments and Analysis}
\label{sec:ea}

This paper uses the test data from the existing literature and we add an extra parameters $K$ into the primary instance in order to meet the demands of research. $K$ means the number of allowed rehandles and it is selected from \{0,10,20,50,100\}. Particular��$K=0$ represents the benchmark and comparison object. There are 80 sets of instances in total, which are categorized by three parameters: the number of ports $P$, which is selected from \{5, 10, 20,30\};  the number of containers $N$, which is selected from \{50,100,200,500,1000,5000\}, and the result isn't shown in the table when $N=5000$; the height limit $H$ selected from \{4,7, 8, 12\}. Each set consists of 5 instances generated by different random seeds. In each instance, the origin $O(i)$ and the destination $D(i)$ of a container $i$ are generated from a uniform distribution on the integers $1,2,\ldots,P$ satisfying that $O(i)<D(i)$. For the convenience of comparison, we only show the average "Numofstack"  when  $K=0$, $K=10$, $K=20$, $K=50$, $K=100$, respectively. And this is applicable to the heuristic algorithm as well as the deuterogenic $P\&H$ algorithm, the simple introduced will be introduced in the next content.

In this section, we will showcase the performance of our heuristic algorithms on a number of test instances. The heuristic was implemented in Java 8 and the experiments were conducted on a computer with Intel Core processor clocked at 2.30 GHz and 8 GB RAM. The operating system of the computer is Windows 10.

The experimental results are shown in Table \ref{tab:1} and Table \ref{tab:2}.

\begin{table}[htbp]
\tiny
  \centering
  \caption{Results of the instances with $P\in\{5,10\}$}
    \begin{tabular}{r|r|r|r|r|r|r|r|r}
    \hline
    $P$     & $N$     & $H$     & UB    & K=0    & K=10     & K=20    & K=50  & K=100  \\
    \hline
    5   & 50    & 4    & 9.6    & 8   & 8     & 8     & 8   & 8     \\
    5   & 50    & 7    & 6.8    & 5   & 5     & 5     & 5   & 5     \\
    5   & 50    & 8    & 6      & 4.6 & 4     & 4     & 4   & 4     \\
    5   & 50    & 12   & 5      & 3.2 & 3     & 3     & 3   & 3     \\
    5   & 100   & 4    & 16.8   & 15  & 15    & 15    & 15  & 15    \\
    5   & 100   & 7    & 10.4   & 9   & 8.8   & 8.8   & 8.8 & 8.8   \\
    5   & 100   & 8    & 9.6    & 7.6 & 7.6   & 7.6   & 7.6 & 7.6   \\
    5   & 100   & 12   & 7.6    & 5.4 & 5.6   & 5.4   & 5.4 & 5.4   \\
    5   & 200   & 4    & 32.2   & 30.6& 30.6  & 30.6  & 30.6& 30.6  \\
    5   & 200   & 7    & 19.4   & 17.6& 17.6  & 17.6  & 17.6& 17.6  \\
    5   & 200   & 8    & 17.4   & 15.6& 15.6  & 15.6  & 15.6& 15.6  \\
    5   & 200   & 12   & 12.4   & 10.4& 10.4  & 10.4  & 10.4& 10.4  \\
    5   & 500   & 4    & 78     &76.6 & 76.6  & 76.6  & 76.6& 76.6  \\
    5   & 500   & 7    & 45.6   &44.2 & 44.2  & 44.2  & 44.2& 44.2  \\
    5   & 500   & 8    & 40     &38.6 & 38.6  & 38.6  & 38.6& 38.6  \\
    5   & 500   & 12   & 40     &27.4 & 26    & 26    & 26  & 26    \\
    5   & 1000  & 4    & 153.8  &152.2& 152.2 & 152.2 & 152.2& 152.2\\
    5   & 1000  & 7    & 89     &87.2 & 87.2  & 87.2  & 87.2& 87.2  \\
    5   & 1000  & 8    & 78.2   &76.6 & 76.6  & 76.6  & 76.6& 76.6 \\
    5   & 1000  & 12   & 52.8   &51   & 51    & 51    & 51  & 51    \\
    10   & 50    & 4    & 12.6   & 9   & 8   & 7.8    & 7.8  &7.8\\
    10   & 50    & 7    & 10     & 6.4 & 6.2 & 5.6    & 5.6  &5.6\\
    10   & 50    & 8    & 10     & 6   & 6   & 5      & 5    &5\\
    10   & 50    & 12   & 9.4    & 5.6 & 4.6 & 4.4    & 4.4  &4.4\\
    10   & 100   & 4    & 19.6   & 15.8& 15.6& 15     & 14.8 &14.8\\
    10   & 100   & 7    & 13.6   & 10.2& 10.4& 10.2   & 8.8  &8.6\\
    10   & 100   & 8    & 12.6   & 9.2 & 9.8 & 9.6    & 7.8  &7.6\\
    10   & 100   & 12   & 10.4   & 7   & 7   & 7.2    & 6.4  &5.2\\
    10   & 200   & 4    & 33.2   & 29.6& 29.6& 28.6   & 28.6 &28.6\\
    10   & 200   & 7    & 21.2   & 17.6& 17.8& 18     & 16.4 &16.4\\
    10   & 200   & 8    & 19.2   & 16.4& 16.6& 16.2   & 14.8 &14.4\\
    10   & 200   & 12   & 15     & 12  & 12  & 11.8   & 11.4 &9.8\\
    10   & 500   & 4    & 74.2   & 69.8& 69.8& 69.8   & 69.8 &69.8\\
    10   & 500   & 7    & 44.4   & 40.6& 40.4& 40.2   & 40   &40\\
    10   & 500   & 8    & 39.4   & 35.6& 35  & 35.2   & 35   &35\\
    10   & 500   & 12   & 28     & 25.2& 24.8& 24.8   & 24   &23.6\\
    10   & 1000  & 4    & 141.8  & 137.6& 137.6& 137.6 & 137.6&137.6\\
    10   & 1000  & 7    & 83.2   & 79   & 79  & 79     & 79   &79\\
    10   & 1000  & 8    & 73.2   & 69   & 69  & 69     & 69   &69\\
    10   & 1000  & 12   & 50.4   & 46.8 & 46.8& 46.6   &46.2  &46.6\\
    \hline
    \end{tabular}
  \label{tab:1}
\end{table}


\begin{table}[htbp]
\tiny
  \centering
  \caption{Results of the instances with $P\in\{20,30\}$}
    \begin{tabular}{r|r|r|r|r|r|r|r|r}
    \hline
    $P$     & $N$     & $H$     & UB    & K=0    & K=10     & K=20    & K=50  & K=100  \\
    \hline
    20   & 50    & 4    & 19.6    & 9   & 9     & 8     & 7,2   & 7.2     \\
    20   & 50    & 7    & 19.2    & 7.8   & 7     & 6.6     & 5.2   &4.4     \\
    20   & 50    & 8    & 19      & 7.8 & 7     & 6.6     & 5.4   & 3.8     \\
    20   & 50    & 12   & 19      & 7.8 & 6.6     &5.6     & 4   & 2.8     \\
    20   & 100   & 4    & 25   & 16  & 16    & 16    & 14.2  & 14.2    \\
    20   & 100   & 7    & 20.4   & 11.6   & 11.4   & 11.2   & 10.6 & 8.4   \\
    20   & 100   & 8    & 19.6    & 10.8 & 11   & 10.8   & 10.2 & 7.6   \\
    20  & 100   & 12   & 19.2    & 10.2 & 9.8   & 9   & 9 & 7.6   \\
    20   & 200   & 4    & 37.6   & 29.8& 30.2  & 29.8  & 28.8& 27.6  \\
    20   & 200   & 7    & 26.8   & 19.2& 20  & 19.4  & 18.8& 18.2  \\
    20   & 200   & 8    & 25.2   & 17.6& 18  & 18  & 17.4& 17.4  \\
    20  & 200   & 12   & 21.6   & 13.8& 13  & 13.8  & 13.2& 13.4  \\
    20   & 500   & 4    & 77.2     &70.2 & 70.4  & 70  & 70.4& 67.6  \\
    20   & 500   & 7    & 48.6   &41.8 & 42.4  & 42.2  & 42.2& 41.8  \\
    20  & 500   & 8    & 43.6     &38 & 37.6  & 37.8  & 38 & 38  \\
    20  & 500   & 12   & 33.2     &27  & 27.2    & 27.2    & 27.2  & 27.2    \\
    20   & 1000  & 4    & 143.6  &135.8 & 135.8 & 136.2 & 136& 134.4\\
    20  & 1000  & 7    & 86.4     &80.4 & 80.6  & 80.6  & 80.6 & 80.4  \\
    20  & 1000  & 8    & 76.8   &70.6 & 71  & 70.4  & 70.4& 70.6 \\
    20  & 1000  & 12   & 54.4   &48.8   & 49.8    & 49.2    & 49.2  & 48.8    \\
    30   & 50    & 4    & 29.8   & 10   & 9.8   & 9.4    & 8  &8\\
    30   & 50    & 7    & 29     & 8.4 & 7.4 & 7.8    & 5.8  &4.8\\
    30    & 50    & 8    & 29     & 8.2   & 7.8   & 7.6      & 5.2   &4.2\\
    30    & 50    & 12   & 29    & 8 & 7.4 & 6.8    & 5.2  &3.4\\
    30    & 100   & 4    & 33.4   & 16.8& 16.6& 16.2    & 15.6 &13.8\\
    30    & 100   & 7    & 29.8   & 12.2& 12& 11.6   & 11.6  &9.4\\
    30   & 100   & 8    & 29.6   & 12 & 11.4 & 11.8    & 10.8  &9.8\\
    30   & 100   & 12   & 29   & 10.8   & 10.4   & 10.2    & 9.8  &8.2\\
    30    & 200   & 4    & 43.8   & 30.4& 31& 31.6   & 31.2 &28\\
    30    & 200   & 7    & 33.8   &20& 19.8& 19.4     & 20.2 &20\\
    30    & 200   & 8    & 32.2   & 18.6& 18.8& 19   & 19.4 &19.8\\
    30   & 200   & 12   & 29.8     & 17  & 16.6  & 16.2  & 16.2 &16.2\\
    30    & 500   & 4    & 80.4   & 69.6& 69.6& 69.6   & 69.4 &69.2\\
    30    & 500   & 7    & 53.4   & 42.4& 42& 42.8   & 42.6   &42.8\\
    30    & 500   & 8    & 49   & 38.4& 38.6  & 38.6   & 38.2   &37.8\\
    30    & 500   & 12   & 39.2     & 29 & 28.6 & 28.6   & 28.6   &28.8\\
    30    & 1000  & 4    & 144  & 133.6& 133.6& 133.6 & 133.8 &133.6\\
    30    & 1000  & 7    & 89.6   & 79.8   & 79.8  & 79.8     & 80   &79.4\\
    30    & 1000  & 8    & 80.4   & 71.4   & 70.6  & 71     & 70.6   &70.6\\
    30   & 1000  & 12   & 59.4   & 50.8 & 50.6& 50.8   &50.8  &50.8\\
    \hline
    \end{tabular}
  \label{tab:2}
\end{table}

Table \ref{tab:1} summarizes the results of instance with small number of ports, i.e. $P \in \{5,10\}$. For each given $P$, $N$, $H$, we work the value $UB$, the average $Numofstack$ when $K$ selects different values. Similarly, the $UB$ is used to test the rationality and  the result is used to compare when $K=0$. Most of instances provide the evidence that it works well when $K$ isn't equal to zero comparing the condition where $K$ is equal to zero.

Table \ref{tab:2} summarizes that the results of the instances with large $P$, i.e. $P \in \{20,30\}$. The columns in this table are similarly defined as those of Table 1 and so is the conclusion we can get from Table 1.


\begin{table}[htbp]
  \centering
  \caption{average results of the instances with heuristic algorithm}
    \begin{tabular}{r|r|r|r|r|r}
    \hline
     $K$       &0   &10  &20  &50  &100\\
    \hline
    $Numofstack$   &97.6854  &97.61042  &97.4875   &97.15  &96.77708\\
 \hline
    \end{tabular}
  \label{tab:3}
\end{table}

Table \ref{tab:3} gives the average results when the $K$ has different values. It isn't difficult to conclude that the value of $Numofstack$ becomes smaller if several rehandle is allowed and the more rehandle, the less stacks used. That's to say, our problem based on zero rehandle is meaningful and our algorithm is practical.




In fact, we have give another related algorithm named $P\&H$ algorithm to get the stowage planning. We give some restrictions on the height and nearport of alternative stacks when choosing one stack for each container. The concrete content of "P\&H" method isn't described in this paper but the computational results will be listed in the following tables. The parameters in $P\&H$ values between 0 and 1. The former parameter represents the dispersion degree between the loading container's destination port and  alternative stack's nearport and their values are equal if the value of parameter is 1. The latter parameter means the  ratio of stack's height and the limited height $H$.

\begin{table}[htbp]
  \centering
  \caption{average results of the instances with $P\&H$ algorithm}
    \begin{tabular}{r|r|r|r|r|r}
    \hline
     $P\&H$       &0   &10  &20  &50  &100\\
    \hline
    $0.75\&0.75$   &97.725  &97.6375  &97.49583   &97.12708  &96.72083\\
    \hline
    $0.75\&0.80$   &97.725  &97.6375  &97.49583   &97.12708  &96.72083\\
    \hline
    $0.80\&0.75$   &97.69583  &97.61667  &97.47083   &97.11667  &96.74583\\
    \hline
    $0.80\&0.80$    &97.69583  &97.61667  &97.47083   &97.11667  &96.74583\\
    \hline
    \end{tabular}
  \label{tab:4}
\end{table}

Table \ref{tab:4} shows the result with "P\&H" method, which is firstly designed to improve the original algorithm and eventually it is used to test our original algorithm as reference. Both the two parameters of the method are selected form \{0.75,0.8\}. We can draw a conclusion that the first parameter has a significant influence on the result while the second parameter makes no difference. I suppose that the it happens as a result of the value set of $H$.  Comparing with the results shown in Table 3, we could find that original algorithm we propose works better when the value of $K$ is small while the $P\&H$ method works better when the value of $K$ is large. So maybe we can use the better one after working out the result when $K$ is certain.

We also use some pictures to make our result clear and the pictures are made by Excel.

\begin{figure}[htbp]
\centering
\includegraphics[width=4.00in,height=3.00in]{K-Stacks.png}
\caption{Relationship between allowed rehandle and used stacks}
\label{fig1:graph}
\end{figure}
As what is shown in Table \ref{tab:3}, Figure \ref{fig1:graph} indicates visually the relationship between used stacks and allowed rehandle. It's not hard to draw the conclusion that the more allowed rehandle, the less used stacks. So it intuitively reflect the meaning of this research topic and it show the feasibility of our algorithm. Comparing with zero rehandle, some rehandle are allowed to reduce the partial stacks and improve the utilization of space.

\begin{figure}[htbp]
\centering
\includegraphics[width=4.00in,height=3.00in]{K-containers.png}
\caption{Relationship between containers and used stacks}
\label{fig2:graph}
\end{figure}
Figure \ref{fig2:graph} shows us the relationship between containers and used stacks. We can conclude that the more containers are, the more stacks will be used, which is undoubted in practice.
\clearpage

\begin{figure}[htbp]
\centering
\includegraphics[width=4.00in,height=3.00in]{K-ports.png}
\caption{Relationship between ports and used stacks}
\label{fig3:graph}
\end{figure}
Figure \ref{fig3:graph} provides us the relationship between ports and used stack. In general, the more ports are among the shipping, the more stacks will be used. Another related factor to be  taken into consideration is the sparse distribution situation of containers in each port.

\begin{figure}[htbp]
\centering
\includegraphics[width=4.00in,height=3.00in]{K-limit height.png}
\caption{Relationship between limit height and used stacks}
\label{fig4:graph}
\end{figure}
Figure \ref{fig4:graph} shows us that the greater the number of limit height is, the less stacks will be used, which is definitely right.


\begin{figure}
\centering
\includegraphics[width=4.00in,height=3.00in]{K-comprehensive.png}
\caption{Relationship between comprehensive factors and used stacks}
\label{fig5:graph}
\end{figure}
Figure \ref{fig5:graph} shows us that the comprehensive impact on the result we work out for all the basic factors considered in this paper. And we can choose the factor we want to analyze and it is convenient for us to work out how different factor influences our  final result. For the convenience of observation, we only give the result with filtered numbers of each factor.


\section{Conclusion}
\label{sec:con}
The stowage stack minimization problem with K-rehandle constraint (SSMP-KR) deals with accommodating give containers throughout a voyage by fewest stacks on a containership subject to K rehandles.
In this paper, we analyze the structure of this combinatorial optimization problem and its relationship with the zero rehandle problem proposed in \cite{wang2014stowage}.
We can regard the stowage stack minimization problem with the zero rehandle as an exception  when we choose the value of K as zero.
Hence, the problem proposed in this paper is a generalized and extended problem considering the existence of rehandle.
A heuristic approach is proposed to construct near-optimal solutions in a very short computational time.
The experimental results show that our heuristic approaches generate very promising solutions on a variety of instances. Comparing with the stowage stack minimum rehandle problem with zero constraint, our problem with K rehandle constraint acts as a buffer and offers great flexibility for the actual problem. The results shows the problem we put forward is of practical significance and it can improve the utilization ratio of vessels with certain flexibility. We use greedy rules to find the best slot for each container, hence the results may be the near-optimal rather than global optimal. As we all know, it is difficult to get the global optimal for this kind of problem. Applying this algorithm into the real shipping management can enhance the hull space utilization and reduce the unnecessary spending in the case of meeting other constraints. The novelty and practicality show the importance of this paper and it can give a reference to the future study.

\bibliographystyle{apalike2}
\bibliography{references}
\end{document}


