\documentclass[a4paper]{article}
\begin{document}

\begin{abstract}
The container ship stowage planning problems(CSSP) has been studied all the time and many approaches have been applied to find out a optimal or near-optimal solution to this problem.
In addition, many relative problems have been proposed and researched from different optimization aspects. 
The stowage stack minimization problem(SSMP) investigates a stowage planning problem when carriers have the obligation to ship all the given containers in different ports, with the objective to utilize the fewest number of stacks on the ship.
The relevant research about SSMP can been seen in \cite{avriel2000container}, \cite{jensen2010complexity}, and \cite{wang2014stowage}, and they focused on the condition that there is no rehandle or shift.
In this paper, we first propose the stowage stack minimization problem with K rehanlds(SSMP-KR) and give the description about it. Next, heuristic algorithm are proposed to construct solutions to the SSMP-KR.
We eventually get the optimal parameters combination for our algorithm through multiple parameter adjustment procedures.
Then, we have discussed and proved the theoretical performance guarantee of the algorithm using the knowledge of inequation and mathematical induction, which proves the feasibility of the algorithm theoretically.
To evaluate the actual performance of our algorithm, we conduct experiments on a set of instances with practical size by Java and compare the results with different values of K , and of course zero is included.
The results demonstrate that our heuristic approaches can generate promising solutions if we allowed some rehandle operations compared with the zero rehandle solutions and random loading solutions.
It is worth mentioning that our problem with K rehandle constraint acts as a buffer and offers great flexibility for the actual problem. 
The results shows the problem we put forward is of practical significance and it can improve the utilization ratio of vessels with certain flexibility. 
Applying this algorithm into the real shipping management can enhance the hull space utilization and reduce the unnecessary spending in the case of meeting other constraints. 
What's more, the novelty and practicality show the importance of this paper and it can give a reference to the future study.

\bibliographystyle{apalike2}
\bibliography{references}

\end{abstract} 
\end{document} 